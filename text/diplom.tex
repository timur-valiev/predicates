\documentclass[a4paper,14pt]{extreport}
\usepackage[utf8]{inputenc}
%\usepackage[cp1251]{inputenc}
\usepackage[english,russian]{babel}
\usepackage[T2A]{fontenc}
\usepackage{amsmath,amssymb,epsfig}
\usepackage{paralist}
\usepackage{indentfirst}
\usepackage[width=17cm,height=24cm]{geometry}
\usepackage[unicode]{hyperref}
\usepackage{cmap}
\usepackage{graphicx}
%\hoffset=20mm
\voffset=8mm
\RequirePackage{enumitem}
%\textwidth=18cm
%\textheight=235mm
%\hoffset=-20mm
%voffset=-15mm
\setenumerate[1]{fullwidth }
\addto\captionsrussian{\def\contentsname{Оглавление}}
\addto\captionsrussian{\def\bibname{Список литературы}}
\makeatletter
\renewcommand{\@biblabel}[1]{#1.\hfil}
%\usepackage{titlesec}
\usepackage[titles]{tocloft}
\renewcommand{\cftchappresnum}{Глава~}
\renewcommand{\cftchapleader}{\bfseries\cftdotfill{\cftdotsep}}
\renewcommand{\cftchapaftersnum}{.}
\newlength{\zyvseclen}
\settowidth{\zyvseclen}{\bfseries\cftchappresnum\cftchapaftersnum}
%\addtolength{\zyvseclen}{2mm}
\addtolength{\cftchapnumwidth}{\zyvseclen}

\newcommand{\proof}{\trivlist \item[\hskip
      \labelsep{\bf Доказательство.}]}

\newtheorem{thm}{Теорема}
\newtheorem{lm}{Лемма}
\newtheorem{sld}{Следствие}
\newtheorem{zam}{Замечание}
\newtheorem{utv}{Утверждение}
\newtheorem{opr}{Определение}
\makeatletter
\def\thmstyle{\it}
\def\@begintheorem#1#2{\it \trivlist \item[\hskip
       \labelsep{\bf #1\ #2.}]\thmstyle}
\def\@opargbegintheorem#1#2#3{\it \trivlist \item[\hskip
       \labelsep{\bf #1\ #2\ (#3).}]\thmstyle}
\makeatother

\newcommand{\normI}{\renewcommand{\baselinestretch}{1.}}
\newcommand{\biggI}{\renewcommand{\baselinestretch}{1.1}}
\biggI

\begin{document}
\newpage
\tableofcontents
\newpage
\chapter{ Введение и постановка задачи }

\section{Введение}
\section{Цели и задачи}
 
\newpage
\chapter{Построение предполных классов четырехзначной логики}
\section{Основные понятия}
\section{Алгоритмы поиска предполных классов}
\section{Результаты}

\newpage
\chapter{Построение таблицы принадлежности функций предполным классам}
\section{Описание алгоритма}
\subsubsection{Перебор функций}
\subsubsection{Проверка сохранения функцией предиката}
\section{Результаты}

\newpage
\chapter{Задача поиска максимального базиса}
Рассмотрим задачу поиска базисов. Особый интерес представляет поиск максимальных или близких к ним по мощности. 

Для поиска решений был разработан переборный вероятностный алгоритм, использующий на входе полученную ранее таблицу принадлежности функций предполным классам и выдающий в бесконечном цикле наборы строк, соответствующие базисам. 

Сформулируем критерий, по которому будет определяться, соответствует ли набор строк таблицы какому-нибудь базису.
Каждой строке таблицы будем сопоставлять функцию(пример), на которой достигается распределение по предикатам описываемое данной строкой. Каждому столбцу ставится в соответствие предполный класс.  
\begin{utv}
Набор строк $R$ таблицы принадлежности функций предполным классам соответствует базису в $P_4$ тогда и только тогда, когда выполнены следующие условия:
\begin{itemize}
\item для каждого столбца найдется строка из $R$, на пересечении которых стоит $0$.    
\item для каждой строки из $R$ найдется столбец такой, что только на пересечении с этой строкой из $R$ в столбце стоит $0$.   
\end{itemize}
Первое условие гарантирует покрытие всех столбцов(предполных классов), второе - невозможность исключения из набора элементов без сохранения первого свойства. 
\end{utv}

\section{Сложность задачи}
Будем обозначать задачу поиска максимального базиса по таблице принадлежности функций предполным классам $A$.

Будем называть набор строк произвольной таблицы, состоящей из нулей и единиц(далее, если не оговорено обратное, будем считать, что все описываемые таблицы состоят из нулей и единиц), базисным, если для него выполнены условия утверждения1.

Будем обозначать задачу поиска максимального по мощности базисного набора таблицы как $A_1$.

Заметим, что задача А сводится к задаче $A_1$.

За $A_0$ обозначим следующую задачу: пусть на вход подается таблица и число $n$. Необходимо выяснить, существует ли базисный набор строк, мощность которого превосходит $n$.

За $A_2$ обозначим следующий частный случай задачи $A_1$: пусть в каждом столбце ровно 2 нуля. Тогда заметим, что таблица, соответствующая такой задаче, будет являться инвертированной матрицей инцидентности для некоторого графа.

За $A_3$ обозначим задачу поиска минимального независимого доминирующего множества в графе.

\begin{lm}
Задача $A_3$ сводится к задаче $A_2$.
\end{lm}
\begin{proof}
Будем называть строку и столбец инцидентными, если на их пересечении стоит 0. 

Будем называть строки смежными, если существует столбец, инцидентный им обеим.

Посмотрим, какими свойствами будет обладать набор $R$ строк, который получится на выходе алгоритма, решающего задачу $A_2$:
\begin{itemize}
\item максимальный по мощности    
\item для каждого столбца найдется хотя бы одна инцидентная ему строка из $R$
\item для каждой строки из $R$ существует столбец, инцидентный ей и никакая другая строка не будет ему инцидентна     
\end{itemize}

Тогда дополнение $Q$ к $R$ будет обладать следующими свойствами:
\begin{itemize}
\item $Q$ минимальное по мощности    
\item никакие две строки из $Q$ не будут смежными
\item для каждой строки из $R$ найдется смежная ей строка из $Q$     
\end{itemize}

Заметим, что таблица, соответствующая задаче $A_2$, будет являться инвертированной матрицей инцидентности для некоторого графа.

Тогда для задачу $A_3$ можно решать следующим образом: построить инвертированную таблицу инцидентности графа, подать на вход алгоритму, решающему задачу $A_2$, взять дополнение к ответу.
\end{proof}

\begin{lm}
Задача $A_2$ сводится к задаче $A_1$.
\end{lm}
\begin{proof}
Это следует из того, что $A_2$ является частным случаем $A_1$.
\end{proof}

\begin{lm}
Задача $A_1$ сводится к задаче $A_0$.
\end{lm}
\begin{proof}
Для того чтобы найти ответ на $A_1$ можно использовать бинарный поиск по мощности базиса и алгоритм для $A_0$.
\end{proof}

\begin{lm}
Задача $A_0$ лежит в классе NP.
\end{lm}
\begin{proof}
Сертификатом будем считать набор строк. Необходимо проверить условия из утверждения. Это можно сделать двумя вложенными циклами, сложность алгоритма будет $O(n*m+n*n)$, где $n$-количество строк,$m$-количество столбцов.
\end{proof}

\begin{thm}
Задача $A_1$ NP-полна.
\end{thm}
\begin{proof}
Из лемм 2 и 3 следует, что задача $A_3$ сводится к $A_1$, что означает, что $A_1$ NP-трудная. Задача $A_1$ сводится к задаче $A_0$, которая лежит в классе NP. Из этих двух фактов, следует, что $A_1$ NP-полна.  
\end{proof}
\section{Алгоритм}
\section{Результаты}


\chapter{Заключение}



\newpage
\addcontentsline{toc}{chapter}{Список литературы}
\begin{thebibliography}{9}
\bibitem{post1}	Post E.L. Introduction to a general theory of elementary propositions // 
Amer. J. Math.— 1921.- V. 43, №4.- P. 163-185. 
\bibitem{yabl} Е.Ю. Захарова, В.Б. Кудрявцев, С.В. Яблонский О предполных в $k$-значных логиках. // ДАН СССР, 1969, т.186, \No 3, стр.509-512 
\bibitem{post2}	Post E.L. Two-valued iterative systems of mathematical logic // Annals of 
Math. Studies. Princeton Univ. Press.— 1941.— V. 5. 
\bibitem{march} Марченков С.С. Функциональные системы с операцией суперпозиции
\bibitem{roz1}	Rosenberg I.G. La structure des fonctions de plusieurs variables sur un ensemble fini // C.R. Acad. Sci. Paris. Ser A.B.— 1965.— V. 260.— P. 3817- 3819. 
\bibitem{roz2}	Rosenberg I.G. Uber die funktionale Vollstandigkeit in den mehrwertigen Logiken // Rozpravy Ceskoslovenske Akad. Ved. Rada Math. Pfir. Ved. Praha.— 1970.— Bd. 80.- S. 3-93. 
\bibitem{git} Адрес проекта в интернете: https://github.com/zloi-timur/predicates 


\end{thebibliography}
\end{document}
