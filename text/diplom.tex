\documentclass[a4paper,14pt]{extreport}
\usepackage[utf8]{inputenc}
%\usepackage[cp1251]{inputenc}
\usepackage[english,russian]{babel}
\usepackage[T2A]{fontenc}
\usepackage{amsmath,amssymb,epsfig}
\usepackage{paralist}
\usepackage{indentfirst}
\usepackage[width=17cm,height=24cm]{geometry}
\usepackage[unicode]{hyperref}
\usepackage{cmap}
\usepackage{graphicx}
%\hoffset=20mm
\voffset=8mm
\RequirePackage{enumitem}
%\textwidth=18cm
%\textheight=235mm
%\hoffset=-20mm
%voffset=-15mm
\setenumerate[1]{fullwidth }
\addto\captionsrussian{\def\contentsname{Оглавление}}
\addto\captionsrussian{\def\bibname{Список литературы}}
\makeatletter
\renewcommand{\@biblabel}[1]{#1.\hfil}
%\usepackage{titlesec}
\usepackage[titles]{tocloft}
\renewcommand{\cftchappresnum}{Глава~}
\renewcommand{\cftchapleader}{\bfseries\cftdotfill{\cftdotsep}}
\renewcommand{\cftchapaftersnum}{.}
\newlength{\zyvseclen}
\settowidth{\zyvseclen}{\bfseries\cftchappresnum\cftchapaftersnum}
%\addtolength{\zyvseclen}{2mm}
\addtolength{\cftchapnumwidth}{\zyvseclen}

\newcommand{\proof}{\trivlist \item[\hskip
      \labelsep{\bf Доказательство.}]}

\newtheorem{thm}{Теорема}
\newtheorem{lm}{Лемма}
\newtheorem{sld}{Следствие}
\newtheorem{zam}{Замечание}
\newtheorem{utv}{Утверждение}
\newtheorem{opr}{Определение}
\makeatletter
\def\thmstyle{\it}
\def\@begintheorem#1#2{\it \trivlist \item[\hskip
       \labelsep{\bf #1\ #2.}]\thmstyle}
\def\@opargbegintheorem#1#2#3{\it \trivlist \item[\hskip
       \labelsep{\bf #1\ #2\ (#3).}]\thmstyle}
\makeatother

\newcommand{\normI}{\renewcommand{\baselinestretch}{1.}}
\newcommand{\biggI}{\renewcommand{\baselinestretch}{1.1}}
\biggI

\begin{document}
\newpage
\tableofcontents
\newpage
\chapter{ Введение и постановка задачи }

\section{Введение}
Исследование систем $k$-значных функций было инициировано Э.Постом~\cite{post1,post2} для сдучая $k=2$.

Все предполные классы в $P_3$ и $P_4$ были описаны С.В.~Яблонским\cite{yabl}

Как известно, каждый предполный класс в $P_k$ можно задать как множество функций, сохраняющих некоторый предикат. Более того, Розенберг показал \cite{roz1, roz2}, что все эти предикаты можно разделить на 6 попарно непересекающихся семейств: $P$, $O$, $L$, $E$, $C$, $B$, описание и исследование которых можно найти в книге Марченкова С.С. \cite{march}. 

\section{Цели и задачи}

\begin{itemize}
\item Разработать програмное средство для работы с функциями и предикатами
\item Выяснить какие из предикатов, задающих предполные классы, имеются в $P_k$, при $k=4$.
\item Для каждой функции двух переменных из $P_4$ определить все предполные классы, которым она принадлежит.
\item Построить множество уникальных строк таблицы принадлежности функций двух переменных предполным классам
\item Исследовать задачу поиска максимального базиса по таблице принадлежности
\end{itemize}

 
\newpage
\chapter{Построение предполных классов четырехзначной логики}
\section{Основные понятия}
Пусть $k$ - натуральное число, $E_k$ - множество чисел от 0 до $k-1$, $P_k$ - множество функций от конечного числа переменных над $E_k$, $P_k^n$ - множество функций от $n$ переменных над $E_k$.  

Определения операции суперпозиции, замыкания, полноты и базиса можно найти в \cite{march}.

Будем называть множество функций замкнутым классом, если оно равно своему замыканию относительно суперпозициию

Пусть $Q, R$ — замкнутые классы и $Q \subset R$ (включение строгое). 
Говорят, что класс $Q$ предполон в классе $R$, если $[Q \cup \{f\}] = R$ для 
любой функции $f$ из множества $R \setminus Q$. 

Предикатом будем называть функцию вида $E_k$ -> {True, False}. Будем задавать предикат множеством векторов, которые ему удовлетворяют(на которых принимается значение True).

Будем говорить, что функция сохраняет предикат, если .....

Как уже было упомянуто, каждый предполный класс в $P_k$ можно задать как множество функций, сохраняющих некоторый предикат. Более того, Розенберг показал \cite{roz1, roz2}, что все эти предикаты можно разделить на 6 попарно непересекающихся семейств: $P$, $O$, $L$, $E$, $C$, $B$. Ниже будет описано каждое из этих семейств и будет предложен алгоритм для поиска и построения предикатов, лежащих в этих семействах.

\section{Алгоритмы поиска предполных классов}
Для решения задачи была написана программа на языке программирования Java, исходный код которой можно найти в \cite{git}. Там же можно найти результаты и вспомогательные материалы, в частности и этот текст.

Архитектура программы, используемой для решения поставленных задач, была построена так, чтобы можно было с минимальными изменениями перейти к работе с функциями и предикатами в $P_k$ при $k > 4$.
 
Для решения поставленной задачи необходимо перебирать однотипные объекты. Для для этой цели хорошо подходят интерфейсы {\tt Iterator} и {\tt Iterable}. Кроме того они позволяют облегчить понимание исходного кода.
При написании программы часто использовались следующие обозначения:
\begin{itemize}
\item$Dim$ – “значность”
\item$Capacity$ – «местность».  
\end{itemize} 

Опишем основные сущности, реализованные в Java классах.

\section{Описание сущностей}
Все определения сущностей находятся в пакете {\tt predicates.domain}
\subsection{Tuple}
Класс, который описывает упорядоченный набор чисел из capacity чисел от $0$ до $dim-1$. реализует интерфейсы {\tt Iterator} и {\tt Iterable}, что позволяет перебирать кортежи по порядку (лексикографическому).
\subsection{Function}
Класс, который описывает функцию, зависящую от $capacity$ переменных чисел, из $P_{dim}$, реализует интерфейсы {\tt Iterator} и {\tt Iterable}, что позволяет перебирать функции по порядку столбцов значений (лексикографическому).
\subsection{Permutation}
Класс, который описывает перестановку чисел от $0$ до $capacity-1$, реализует интерфейсы {\tt Iterator} и {\tt Iterable}, что позволяет перебирать перестановки по порядку (лексикографическому).
\subsection{Predicate} 
Описывает предикат, используя {\tt Set<ImmutableList<Integer> >} - множество векторов, удовлетворяющих предикату.

Для каждого семейства предикатов был написан Java-класс \\{\tt PredicateFactory\_X}, где X – название соответствующего семейства. Каждый из них реализует интерфейсы {\tt Iterable<Predicate>}, {\tt Iterator<Predicate>}.

Все они лежат в пакете {\tt predicates.factory}.

Т.к. при нахождении необходимых предикатов во многих случаях требуется перебор нескольких параметров и при некоторых комбинациях могут получаться  одинаковые результаты, необходимо было не выдавать уже полученные идентичные предикаты. Эта  проблема чаще всего решалась использованием структур данных, реализующих интерфейс {\tt Set<Predicate>}. 

Перейдем к подробному описанию каждого из модулей программы, соотвествующих семействам предикатов.

\subsection{Семейство $P$}
\begin{opr}
Семейство Р. Предикаты этого семейства существуют при любом $k \leqslant 2$. Пусть $\pi$ ---перестановка на $E_k$, которая разлагается в произведение циклов одной и той же простой длины. Для любой такой перестановки $\pi$ в семейство $P$ входит предикат $\pi(x_1)=x_2$, который называется графиком перестановки $\pi$. 
\end{opr}

Каждый из предикатов класса $Р$ задается перестановкой, которая является произведением циклов одной и той же простой длины (при $k=4$ получается $2$ цикла длины $2$), поэтому для нахождения всех предикатов программа перебирает все перестановки, проверяет на выполнение вышеописанного условия (в классе {\tt Permutation} есть соответствующий метод) и, при успешном результате проверки, строит предикат.

\subsection{Семейство $O$}
\begin{opr}
Семейство $O$ содержит любой двуместный предикат, который задает на $E_k$ частичный порядок с
наименьшим и наибольшим элементами (ограниченный частичный порядок). 
\end{opr}

Для нахождения всех таких предикатов программа перебирает все перестановки чисел от $0$ до $dim-1$ (от $0$ до $3$ при $k=4$), считая, что первый элемент перестановки будет наибольшим, последний – наименьшим. 

Далее берутся все пары элементов $(a_i,a_j)$, где $1<i<j<k$ и перебираются все из $2^{(k-2)*(k-3)/2}$ вариантов считать или не считать, что эта пара пренадлежит предикату, причем отсекаются те конфигурации, не обладающие свойством транзитивности. Для каждого подошедшего варианта строится предикат.

\subsection{Семейство $E$}
\begin{opr}
Семейство $Е$ состоит из всех двуместных предикатов, которые  представляют собой отношения эквивалентности на $E_{k}$, отличные от полного и единичного отношений (нетривиальные отношения эквивалентности). Таким образом, каждое отношение эквивалентности из $Е$ разбивает множество $E_{k}$ на $l$ классов попарно эквивалентных элементов, где $1 < l < k$. 
\end{opr}
Нахождение всех таких предикатов осуществляется при помощи перебора всех возможных разбиений и построения соответствующих предикатов. 

\subsection{Семейство $L$}
\begin{opr}
Предикаты этого семейства существуют только при $k = p^l$, где $p$ — простое число и $l > 0$. В этом случае на множестве $E_k$ можно определить бинарную коммутативную операцию $+$ так, 
что $G = <E_k;+>$ будет являться абелевой $p$-группой периода $p$. Иными словами, в абелевой группе $G$ порядок любого элемента, отличного от нуля группы, равен $p$. 
\end{opr}
Итак, если $k = p^l$ и $G = <E_k; +>$ — абелева $p$-группа периода $p$, то семейству $L$ принадлежит предикат $x_1+x_2=x_3+x_4$. 

При $k=4$ и $p=l=2$ существует только одна таблица сложения, удовлетворяющая описанным условиям: 
\begin{center}
\begin{tabular}{|c|c|c|c|c|}
\hline
&$x_1$&$x_2$&$x_3$&$x_4$\\
\hline
$x_1$&0&1&2&3\\
\hline
$x_2$&1&0&3&2\\
\hline
$x_3$&2&3&0&1\\
\hline
$x_4$&3&2&1&0\\
\hline
\end{tabular}
\end{center}

Для нахождения всех таких предикатов программа перебирает все перестановки чисел от $0$ до $dim-1$ (от $0$ до $3$ при $k=4$) и строит предикат $x_1+x_2=x_3+x_4$. При $k=4$ имеется только предикат в этом семействе.
\subsection{Семейство $C$}
\begin{opr}
Семейство $C$ состоит из всех центральных предикатов.
\end{opr}

\begin{opr} Предикат $p(x_1,...,x_m )$ называется центральным, если он вполне рефлексивен, вполне симметричен, отличен от тождественно истинного предиката и существует такое непустое подмножество $C$ множества $E_k$ (центр предиката $p$),  что предикату $p$ удовлетворяет всякий набор $(a_1,..., a_m)$ из $E_k^m$, как только $(a_1,..., a_m) \cap C \neq \emptyset$.
\end{opr}
\begin{opr}Предикат $p(x_1,..., x_m)$ называется вполне рефлексивным, если либо $m = 1$, либо  если $m > 1$ и $p(a_1,..., a_m) = True$ для любого набора $(a_1,...,a_m)$ из $E_k^m$, содержащего не более $m-1$ различных значений.
\end{opr}
\begin{opr} 
Предикат $p$ называется вполне симметричным, если он не меняется при любой перестановке переменных.  
\end{opr}

Для перебора всех предикатов нам необходимо перебрать размерность предиката.

Далее, для каждой размерности, мы строим минимальный вполне рефлексивный предикат.
На следующем шаге мы перебираем все возможные центры и добавляем вектора, которые имеют с ним непустое пересечение, в предикат.

На последнем этапе перебора мы  всеми возможными способами пытаемся его расширить еще не вошедшими векторами, учитывая условия симметричности и нетривиальности.

\subsection{Семейство $B$}
\begin{opr}
Для любого $m \geqslant 2$ положим
$$ \tau_m(x_1, \ldots, x_m) = \bigvee_{1\leqslant i < j \leqslant m}(x_i=x_j) $$
\end{opr}
\begin{opr}
Если $h \geqslant 3, l \geqslant 1, k \geqslant h^l$ и $q$ — отображение множества $E_k$ на  множество $E_{h^l}$, то семейству $B$ принадлежит предикат, который является полным прообразом $l$-й декартовой степени предиката $\tau_h$ при отображении $q$.
\end{opr}
При $k=4$ нам подходит $h=3,4$ и $l=1$.

Для каждого из двух вариантов построим множества предикатов следующим образом:
\begin{enumerate}
\item Переберем все отображения множества $E_k$ на  множество $E_{h^l}$.
\item Для каждого отображения будем строить предикат answer, перебирая все возможные кортежи и проверяя, должны ли они входить в него при условии того, что answer должен быть прообразом предиката $\tau$.
\end{enumerate} 

\section{Результаты}
Построены и сохранены все предикаты, соотвествуюшие предполным классам в $P_4$.  


\newpage
\chapter{Построение таблицы принадлежности функций предполным классам}
Получив все предикаты, описывающие предполные классы, можно перейти к построению таблицы распределения функций двух переменных по предполным классам. Нас будут интересовать уникальные строки этой таблицы. 

\section{Описание алгоритма}
Для решения задачи был разработан переборный алгоритм, ипользующий отсечения. Его основной частью является цикл, перебирающий функции, в котором для каждой из фукций происходит проверка на пренадлежность предикатам.
\subsubsection{Структуры данных}
Для того, чтобы повысить быстродействие при работе с объектами, хранящими информацию о функциях и предикатах, было решено использовать упростить эти структуры. Это было достигнуто использованием стандартных массивов вместо списков и множеств, а так же использованием примитивных типов.

Были реализованы методы преобразования объектов из прошлой версии в новые, при работе которых совершаются предподсчеты, необходимые для более быстрой работы алгоритмов:
\begin{enumerate}
\item Для каждого предиката храним массив булевых переменных, $i$ элемент которого говорит нам о том, что вектор с кодом $i$ принадлежит предикату. 
\item Для каждого предиката храним массив кодов всех пар векторов, принадлежащих ему, и для каждого возможного кода в массиве храним, принадлежит ли он предикату.
\item Для каждой функции храним массив значений. 
\item Для каждой функции храним массив булевых значений, $i$ элемент которого указывает на то, может ли функция принимать значение $i$. 
\item Заводим дополнительный массив, в котором для каждого вектора пар аргументов будем сохранять вычисленные значения в виде кода вектора. 
\end{enumerate} 

\subsubsection{Перебор функций}
\begin{utv}
Если функция $f(x, y)$ сохраняет предикат $P$, то и функция $g(x, y) \equiv f(y, x)$ сохраняет его.
\end{utv}
Это утверждение следует из определения сохранения функцией предиката.

При помощи этого утверждения можно оптимизировать перебор следующим образом: если мы уже рассматривали функцию, которая сохраняет те же предикаты, то можно сразу перейти к следующему шагу цикла.

Если перебирать функции в лексикографическом порядке, то проверить условие досрочного перехода на следующий шаг цикла можно так: нужно взять пару аргументов $(i, j)$ такую, что код этой пары максимален, при том, что $i > j$ и $f(i, j) \neq f(j, i)$. Если такая пара нашлась, то можно не проверять функцию, если $f(i, j) > f(j, i)$.

\subsubsection{Проверка сохранения функцией предиката}
Для проверки сохранения функцией предиката из семейства $B$ при $k = 4$ (существует единственный такой предикат) был введен отдельный алгоритм, в основе которого лежит следующее утверждение.

\begin{utv}
Если для функции $f(x, y)$ найдутся два вектора $(x_1, x_2, x_3, x_4), (y_1, y_2, y_3, y_4), x_i<4, y_i<4 $, таких, что в каждом найдется пара одинаковых чисел, и в векторе$(f(x_1, y_1),f(x_2, y_2),f(x_3, y_3),f(x_4, y_4))$ все элементы попарно различны, то функция $f$ не сохраняет предикат.  
\end{utv}

В алгоритме строятся такие вектора и в случае успеха, функция проверки возвращает $false$, иначе $true$.

Для всех остальных предикатов используется следующий алгоритм: перебираются коды всех пар векторов из предиката(они заранее предподсчитаны) и производится проверка, если ли среди векторов предиката тот, который возвращает функция. Если ответ отрицательный, то происходит возврат из функции со значением $false$. Если для всех пар векторов значение функции на них принадлежит предикату, то возвращается $true$. 

\section{Результаты}
Были получены 122283 уникальных строк таблицы принадлежности функций предполным классам.


\newpage
\chapter{Задача поиска максимального базиса}
Рассмотрим задачу поиска базисов. Особый интерес представляет поиск максимальных или близких к ним по мощности. 

Для поиска решений был разработан переборный вероятностный алгоритм, использующий на входе полученную ранее таблицу принадлежности функций предполным классам и выдающий в бесконечном цикле наборы строк, соответствующие базисам. 

Сформулируем критерий, по которому будет определяться, соответствует ли набор строк таблицы какому-нибудь базису.
Каждой строке таблицы будем сопоставлять функцию(пример), на которой достигается распределение по предикатам описываемое данной строкой. Каждому столбцу ставится в соответствие предполный класс.  
\begin{utv}
Набор строк $R$ таблицы принадлежности функций предполным классам соответствует базису в $P_4$ тогда и только тогда, когда выполнены следующие условия:
\begin{itemize}
\item для каждого столбца найдется строка из $R$, на пересечении которых стоит $0$.    
\item для каждой строки из $R$ найдется столбец такой, что только на пересечении с этой строкой из $R$ в столбце стоит $0$.   
\end{itemize}
Первое условие гарантирует покрытие всех столбцов(предполных классов), второе - невозможность исключения из набора элементов без сохранения первого свойства. 
\end{utv}

\section{Сложность задачи}
Будем обозначать задачу поиска максимального базиса по таблице принадлежности функций предполным классам $A$.

Будем называть набор строк произвольной таблицы, состоящей из нулей и единиц(далее, если не оговорено обратное, будем считать, что все описываемые таблицы состоят из нулей и единиц), базисным, если для него выполнены условия утверждения1.

Будем обозначать задачу поиска максимального по мощности базисного набора таблицы как $A_1$.

Заметим, что задача А сводится к задаче $A_1$, так как является ее частным случаем.

За $A_0$ обозначим следующую задачу: пусть на вход подается таблица и число $n$. Необходимо выяснить, существует ли базисный набор строк, мощность которого превосходит $n$.

За $A_2$ обозначим следующий частный случай задачи $A_1$: пусть в каждом столбце ровно 2 нуля. Тогда заметим, что таблица, соответствующая такой задаче, будет являться инвертированной матрицей инцидентности для некоторого графа.

За $A_3$ обозначим задачу поиска минимального независимого доминирующего множества в графе.

Будем называть строку и столбец инцидентными, если на их пересечении стоит 0. 

Будем называть строки смежными, если существует столбец, инцидентный им обеим.

\begin{lm}
Задача $A_3$ сводится к задаче $A_2$.
\end{lm}
\begin{proof}
Посмотрим, какими свойствами будет обладать набор $R$ строк, который получится на выходе алгоритма, решающего задачу $A_2$:
\begin{itemize}
\item максимальный по мощности    
\item для каждого столбца найдется хотя бы одна инцидентная ему строка из $R$
\item для каждой строки из $R$ существует столбец, инцидентный ей и никакая другая строка не будет ему инцидентна     
\end{itemize}

Тогда дополнение $Q$ к $R$ будет обладать следующими свойствами:
\begin{itemize}
\item $Q$ минимальное по мощности    
\item никакие две строки из $Q$ не будут смежными
\item для каждой строки из $R$ найдется смежная ей строка из $Q$     
\end{itemize}

Заметим, что таблица, соответствующая задаче $A_2$, будет являться инвертированной матрицей инцидентности для некоторого графа.

Тогда для задачу $A_3$ можно решать следующим образом: построить инвертированную таблицу инцидентности графа, подать на вход алгоритму, решающему задачу $A_2$, взять дополнение к ответу.
\end{proof}

\begin{lm}
Задача $A_2$ сводится к задаче $A_1$.
\end{lm}
\begin{proof}
Это следует из того, что $A_2$ является частным случаем $A_1$.
\end{proof}

\begin{lm}
Задача $A_1$ сводится к задаче $A_0$.
\end{lm}
\begin{proof}
Для того чтобы найти ответ на $A_1$ можно использовать бинарный поиск по мощности базиса и алгоритм для $A_0$.
\end{proof}

\begin{lm}
Задача $A_0$ лежит в классе NP.
\end{lm}
\begin{proof}
Сертификатом будем считать набор строк. Необходимо проверить условия из утверждения. Это можно сделать двумя вложенными циклами, сложность алгоритма будет $O(n*m+n*n)$, где $n$-количество строк,$m$-количество столбцов.
\end{proof}

\begin{thm}
Задача $A_1$ NP-полна.
\end{thm}
\begin{proof}
Из лемм 2 и 3 следует, что задача $A_3$ сводится к $A_1$, что означает, что $A_1$ NP-трудная. Задача $A_1$ сводится к задаче $A_0$, которая лежит в классе NP. Из этих двух фактов, следует, что $A_1$ NP-полна.  
\end{proof}
\section{Алгоритм}
Был разработан вероятностный алгоритм решения задачи $A_1$. На вход ему подавалась таблица, состоящая из уникальных строк таблицы принадлежности функций предполным классам, что позволило искать нижние оценки для мощности максимального базиса $P_4$.

Алгоритм можно разелить на две основные части: подготовка структур данных, которые будут использоваться дальше, по входной таблице(предподсчет) и основной цикл(бесконечный), в котором строятся наборы строк и производится проверка на корректность и обновление максимальных результатов, если есть такая необходимость. 

\subsection{Подготовка структур данных}
Построим и будем хранить в памяти следующие данные:
\begin{itemize}
\item для каждой строки будем хранить список инцидентных ей стоблцов
\item для каждого столбца будем хранить список инцидентных ему строк
\item для каждого стобца $j$ и строки $i$ будем хранить количество столбцов, инцидентных $i$, номера которых больше $j$ 

По этим данным для каждого столбца $i$ построим объект класса WeightRandGenerator, у которого реалзован метод, который по входному параметру $need$ возвращает случайную функцию, у которой по крайней мере $need$ инцидентных ей столбцов, начиная с $i$.
\end{itemize}
\subsection{Итерация сновного цикла}
Будем строить набор строк итеративно. В начальный момент нам известен текущий максимальный базисный набор и его можность $m$. В каждый момент времени будем хранить текущий полученый набор(обозначим его за $result$), множество столбцов $usedPredicates$, которым инцидентна хотя бы одна строка из $result$.

Будем перебирать столбцы. Если очередной столбец уже есть в $usedPredicates$, то переходим к следующей итерации, иначе добавляем к $result$ функцию, которая определяется методом nextFunction() и обновляем множества $result$ и $usedPredicates$.

После построения производится проверка набора строк(является ли он базисным) и, при необходимости, обновляются максимальные результаты.
 
\section{Результаты}
Была получена нижняя оценка на мощность максимального базиса в $P_4$ --- 19.


\chapter{Заключение}



\newpage
\addcontentsline{toc}{chapter}{Список литературы}
\begin{thebibliography}{9}
\bibitem{post1}	Post E.L. Introduction to a general theory of elementary propositions // 
Amer. J. Math.— 1921.- V. 43, №4.- P. 163-185. 
\bibitem{yabl} Е.Ю. Захарова, В.Б. Кудрявцев, С.В. Яблонский О предполных в $k$-значных логиках. // ДАН СССР, 1969, т.186, \No 3, стр.509-512 
\bibitem{post2}	Post E.L. Two-valued iterative systems of mathematical logic // Annals of 
Math. Studies. Princeton Univ. Press.— 1941.— V. 5. 
\bibitem{march} Марченков С.С. Функциональные системы с операцией суперпозиции
\bibitem{roz1}	Rosenberg I.G. La structure des fonctions de plusieurs variables sur un ensemble fini // C.R. Acad. Sci. Paris. Ser A.B.— 1965.— V. 260.— P. 3817- 3819. 
\bibitem{roz2}	Rosenberg I.G. Uber die funktionale Vollstandigkeit in den mehrwertigen Logiken // Rozpravy Ceskoslovenske Akad. Ved. Rada Math. Pfir. Ved. Praha.— 1970.— Bd. 80.- S. 3-93. 
\bibitem{git} Адрес проекта в интернете: https://github.com/zloi-timur/predicates 


\end{thebibliography}
\end{document}
