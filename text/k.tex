\documentclass[a4paper,14pt]{extreport}
\usepackage[utf8]{inputenc}
%\usepackage[cp1251]{inputenc}
\usepackage[english,russian]{babel}
\usepackage[T2A]{fontenc}
\usepackage{amsmath,amssymb,epsfig}
\usepackage{paralist}
\usepackage{indentfirst}
\usepackage[width=17cm,height=24cm]{geometry}
\usepackage[unicode]{hyperref}
\usepackage{cmap}
\usepackage{graphicx}
%\hoffset=20mm
\voffset=8mm
\RequirePackage{enumitem}
%\textwidth=18cm
%\textheight=235mm
%\hoffset=-20mm
%voffset=-15mm
\setenumerate[1]{fullwidth }
\addto\captionsrussian{\def\contentsname{Оглавление}}
\addto\captionsrussian{\def\bibname{Список литературы}}
\makeatletter
\renewcommand{\@biblabel}[1]{#1.\hfil}
%\usepackage{titlesec}
\usepackage[titles]{tocloft}
\renewcommand{\cftchappresnum}{Глава~}
\renewcommand{\cftchapleader}{\bfseries\cftdotfill{\cftdotsep}}
\renewcommand{\cftchapaftersnum}{.}
\newlength{\zyvseclen}
\settowidth{\zyvseclen}{\bfseries\cftchappresnum\cftchapaftersnum}
%\addtolength{\zyvseclen}{2mm}
\addtolength{\cftchapnumwidth}{\zyvseclen}

\newcommand{\proof}{\trivlist \item[\hskip
      \labelsep{\bf Доказательство.}]}

\newtheorem{thm}{Теорема}
\newtheorem{lm}{Лемма}
\newtheorem{sld}{Следствие}
\newtheorem{zam}{Замечание}
\newtheorem{utv}{Утверждение}
\newtheorem{opr}{Определение}
\makeatletter
\def\thmstyle{\it}
\def\@begintheorem#1#2{\it \trivlist \item[\hskip
       \labelsep{\bf #1\ #2.}]\thmstyle}
\def\@opargbegintheorem#1#2#3{\it \trivlist \item[\hskip
       \labelsep{\bf #1\ #2\ (#3).}]\thmstyle}
\makeatother

\newcommand{\normI}{\renewcommand{\baselinestretch}{1.}}
\newcommand{\biggI}{\renewcommand{\baselinestretch}{1.1}}
\biggI

\begin{document}
\newpage
\tableofcontents
\newpage
\chapter{ Введение и постановка задачи }

\section{Введение}
Исследование систем $k$-значных функций было инициировано Э.Постом~\cite{post1,post2} для сдучая $k=2$.

Все предполные классы в $P_3$ и $P_4$ были описаны С.В.~Яблонским\cite{yabl}

Как известно, каждый предполный класс в $P_k$ можно задать как множество функций, сохраняющих некоторый предикат. Более того, Розенберг показал \cite{roz1, roz2}, что все эти предикаты можно разделить на 6 попарно непересекающихся семейств: $P$, $O$, $L$, $E$, $C$, $B$, описание которых можно найти в книге Марченкова С.С. \cite{march}. 

\section{Постановка задачи}
Пусть $Q, R$ — замкнутые классы и $Q \subset R$ (включение строгое). 
Говорят, что класс $Q$ предполон в классе $R$, если $[Q \cup \{f\}] = R$ для 
любой функции $f$ из множества $R \setminus Q$. 

\begin{itemize}
\item Выяснить какие из предикатов, задающих предполные классы, имеются в $P_k$, при $k=4$.
\item Разработать программный продукт, позволяющий работать с предикатами и функциями, в частности, для заданного конечного множества функций в $P_k$ предоставлять возможность проверки факта сохранения функцией предиката.
\item Для каждой функции одной переменной из $P_4$ определить все предполные классы, которым она принадлежит.
\end{itemize}
 
\newpage
\chapter{Описание программы}
Для решения задачи была написана программа на языке программирования Java, исходный код которой можно найти в \cite{git}. Там же можно найти результаты и вспомогательные материалы, в частности и этот текст.

Для создания проекта использовалась методология {\tt DDD (Domain-driven design)}. 

Для решения поставленной задачи необходимо перебирать однотипные объекты. Для для этой цели хорошо подходят интерфейсы {\tt Iterator} и {\tt Iterable}. Кроме того они позволяют облегчить понимание исходного кода.
При написании программы часто использовались следующие обозначения:
\begin{itemize}
\item$Dim$ – “значность”
\item$Capacity$ – «местность».  
\end{itemize} 

Опишем основные сущности, реализованные в Java классах.

\section{Описание сущностей}
Все определения сущностей находятся в пакете {\tt predicates.domain}
\subsection{Tuple}
Класс, который описывает упорядоченный набор чисел из capacity чисел от $0$ до $dim-1$. реализует интерфейсы {\tt Iterator} и {\tt Iterable}, что позволяет перебирать кортежи по порядку (лексикографическому).
\subsection{Function}
Класс, который описывает функцию, зависящую от $capacity$ переменных чисел, из $P_{dim}$, реализует интерфейсы {\tt Iterator} и {\tt Iterable}, что позволяет перебирать функции по порядку столбцов значений (лексикографическому).
\subsection{Permutation}
Класс, который описывает перестановку чисел от $0$ до $capacity-1$, реализует интерфейсы {\tt Iterator} и {\tt Iterable}, что позволяет перебирать перестановки по порядку (лексикографическому).
\subsection{Predicate} 
Описывает предикат, используя {\tt Set<ImmutableList<Integer> >} - множество векторов, удовлетворяющих предикату.

\section{Описание алгоритмов нахождения и перебора предикатов}
Для каждого семейства предикатов был написан Java-класс \\{\tt PredicateFactory\_X}, где X – название соответствующего семейства. Каждый из них реализует интерфейсы {\tt Iterable<Predicate>}, {\tt Iterator<Predicate>}.

Все они лежат в пакете {\tt predicates.factory}.

Т.к. при нахождении необходимых предикатов во многих случаях требуется перебор нескольких параметров и при некоторых комбинациях могут получаться  одинаковые результаты, необходимо было не выдавать уже полученные идентичные предикаты. Эта  проблема чаще всего решалась использованием структур данных, реализующих интерфейс {\tt Set<Predicate>}. 

Перейдем к подробному описанию каждого из модулей программы, соотвествующих семействам предикатов.

\subsection{Семейство $P$}
\begin{opr}
Семейство Р. Предикаты этого семейства существуют при любом $k \leqslant 2$. Пусть $\pi$ ---перестановка на $E_k$, которая разлагается в произведение циклов одной и той же простой длины. Для любой такой перестановки $\pi$ в семейство $P$ входит предикат $\pi(x_1)=x_2$, который называется графиком перестановки $\pi$. 
\end{opr}

Каждый из предикатов класса $Р$ задается перестановкой, которая является произведением циклов одной и той же простой длины (при $k=4$ получается $2$ цикла длины $2$), поэтому для нахождения всех предикатов программа перебирает все перестановки, проверяет на выполнение вышеописанного условия (в классе {\tt Permutation} есть соответствующий метод) и, при успешном результате проверки, строит предикат.

\subsection{Семейство $O$}
\begin{opr}
Семейство $O$ содержит любой двуместный предикат, который задает на $E_k$ частичный порядок с
наименьшим и наибольшим элементами (ограниченный частичный порядок). 
\end{opr}

Для нахождения всех таких предикатов программа перебирает все перестановки чисел от $0$ до $dim-1$ (от $0$ до $3$ при $k=4$), считая, что первый элемент перестановки будет наибольшим, последний – наименьшим. 

Далее берутся все пары элементов $(a_i,a_j)$, где $1<i<j<k$ и перебираются все из $2^{(k-2)*(k-3)/2}$ вариантов считать или не считать, что эта пара пренадлежит предикату, причем отсекаются те конфигурации, не обладающие свойством транзитивности. Для каждого подошедшего варианта строится предикат.

\subsection{Семейство $E$}
\begin{opr}
Семейство $Е$ состоит из всех двуместных предикатов, которые  представляют собой отношения эквивалентности на $E_{k}$, отличные от полного и единичного отношений (нетривиальные отношения эквивалентности). Таким образом, каждое отношение эквивалентности из $Е$ разбивает множество $E_{k}$ на $l$ классов попарно эквивалентных элементов, где $1 < l < k$. 
\end{opr}
Нахождение всех таких предикатов осуществляется при помощи перебора всех возможных разбиений и построения соответствующих предикатов. 

\subsection{Семейство $L$}
\begin{opr}
Предикаты этого семейства существуют только при $k = p^l$, где $p$ — простое число и $l > 0$. В этом случае на множестве $E_k$ можно определить бинарную коммутативную операцию $+$ так, 
что $G = <E_k;+>$ будет являться абелевой $p$-группой периода $p$. Иными словами, в абелевой группе $G$ порядок любого элемента, отличного от нуля группы, равен $p$. 
\end{opr}
Итак, если $k = p^l$ и $G = <E_k; +>$ — абелева $p$-группа периода $p$, то семейству $L$ принадлежит предикат $x_1+x_2=x_3+x_4$. 

При $k=4$ и $p=l=2$ существует только одна таблица сложения, удовлетворяющая описанным условиям: 
\begin{center}
\begin{tabular}{|c|c|c|c|c|}
\hline
&$x_1$&$x_2$&$x_3$&$x_4$\\
\hline
$x_1$&0&1&2&3\\
\hline
$x_2$&1&0&3&2\\
\hline
$x_3$&2&3&0&1\\
\hline
$x_4$&3&2&1&0\\
\hline
\end{tabular}
\end{center}

Для нахождения всех таких предикатов программа перебирает все перестановки чисел от $0$ до $dim-1$ (от $0$ до $3$ при $k=4$) и строит предикат $x_1+x_2=x_3+x_4$. При $k=4$ имеется только предикат в этом семействе.
\subsection{Семейство $C$}
\begin{opr}
Семейство $C$ состоит из всех центральных предикатов.
\end{opr}

\begin{opr} Предикат $p(x_1,...,x_m )$ называется центральным, если он вполне рефлексивен, вполне симметричен, отличен от тождественно истинного предиката и существует такое непустое подмножество $C$ множества $E_k$ (центр предиката $p$),  что предикату $p$ удовлетворяет всякий набор $(a_1,..., a_m)$ из $E_k^m$, как только $(a_1,..., a_m) \cap C \neq \emptyset$.
\end{opr}
\begin{opr}Предикат $p(x_1,..., x_m)$ называется вполне рефлексивным, если либо $m = 1$, либо  если $m > 1$ и $p(a_1,..., a_m) = True$ для любого набора $(a_1,...,a_m)$ из $E_k^m$, содержащего не более $m-1$ различных значений.
\end{opr}
\begin{opr} 
Предикат $p$ называется вполне симметричным, если он не меняется при любой перестановке переменных.  
\end{opr}

Для перебора всех предикатов нам необходимо перебрать размерность предиката.

Далее, для каждой размерности, мы строим минимальный вполне рефлексивный предикат.
На следующем шаге мы перебираем все возможные центры и добавляем вектора, которые имеют с ним непустое пересечение, в предикат.

На последнем этапе перебора мы  всеми возможными способами пытаемся его расширить еще не вошедшими векторами, учитывая условия симметричности и нетривиальности.

\subsection{Семейство $B$}
\begin{opr}
Для любого $m \geqslant 2$ положим
$$ \tau_m(x_1, \ldots, x_m) = \bigvee_{1\leqslant i < j \leqslant m}(x_i=x_j) $$
\end{opr}
\begin{opr}
Если $h \geqslant 3, l \geqslant 1, k \geqslant h^l$ и $q$ — отображение множества $E_k$ на  множество $E_{h^l}$, то семейству $B$ принадлежит предикат, который является полным прообразом $l$-й декартовой степени предиката $\tau_h$ при отображении $q$.
\end{opr}
При $k=4$ нам подходит $h=3,4$ и $l=1$.

Для каждого из двух вариантов построим множества предикатов следующим образом:
\begin{enumerate}
\item Переберем все отображения множества $E_k$ на  множество $E_{h^l}$.
\item Для каждого отображения будем строить предикат answer, перебирая все возможные кортежи и проверяя, должны ли они входить в него при условии того, что answer должен быть прообразом предиката $\tau$.
\end{enumerate} 

\section{Получение результатов}
Теперь, получив все искомые предикаты, необходимо проверить, какие функции сохраняют их. Для такой проверки в проекте есть класс PredicateService, который содержит метод с сигнатурой \\{\tt public static boolean checkSave (Predicate predicate, Function function)}. 

Теперь достаточно перебрать все пары, состоящие из функций и найденных предикатов, и применить к ним данный метод.

\newpage
\chapter{Результаты и заключение}
\begin{enumerate}
\item Были найдены все предикаты, описывающие предполные классы в $P_4$, и построена таблица (см. приложение А) распределения функций одной переменной четырехзначной логики этим предикатам. 
\item Была создан Java-проект с архитектурой, позволяющей использовать его для задач, связанных с предикатами и предполными классами при значениях $k \geqslant 3$ (сейчас только при нахождении предикатов семейств $L$ и $B$ есть ограничения на $k$).
Исходный код программы находится в приложении Б.
\end{enumerate}

\newpage
\addcontentsline{toc}{chapter}{Список литературы}
\begin{thebibliography}{9}
\bibitem{post1}	Post E.L. Introduction to a general theory of elementary propositions // 
Amer. J. Math.— 1921.- V. 43, №4.- P. 163-185. 
\bibitem{yabl} Е.Ю. Захарова, В.Б. Кудрявцев, С.В. Яблонский О предполных в $k$-значных логиках. // ДАН СССР, 1969, т.186, \No 3, стр.509-512 
\bibitem{post2}	Post E.L. Two-valued iterative systems of mathematical logic // Annals of 
Math. Studies. Princeton Univ. Press.— 1941.— V. 5. 
\bibitem{march} Марченков С.С. Функциональные системы с операцией суперпозиции
\bibitem{roz1}	Rosenberg I.G. La structure des fonctions de plusieurs variables sur un ensemble fini // C.R. Acad. Sci. Paris. Ser A.B.— 1965.— V. 260.— P. 3817- 3819. 
\bibitem{roz2}	Rosenberg I.G. Uber die funktionale Vollstandigkeit in den mehrwertigen Logiken // Rozpravy Ceskoslovenske Akad. Ved. Rada Math. Pfir. Ved. Praha.— 1970.— Bd. 80.- S. 3-93. 
\bibitem{git} Адрес проекта в интернете: https://github.com/zloi-timur/predicates 


\end{thebibliography}
\end{document}
